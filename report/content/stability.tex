\subsection{Устойчивость верхнего положения равновесия}

Проверим, когда положение равновесия для неподвижной тележки с координатой $s(t) \equiv s^*$ является неустойчивым.
Так как $s(t) = \mathrm{const}$, то $\dot s(t) = \ddot s(t) \equiv 0$, тогда уравнение~\eqref{eq:modeling_equation} примет вид:
\begin{equation}\label{eq:stability_1}
	\ddot \varphi = - \frac{\xi}{ml}\varphi + \frac{g}{l}\sin\varphi
\end{equation}
Теперь воспользуемся следующей теоремой, доказанной в курсе <<Теории устойчивости>>:
\begin{theorem}[Ляпунов]
	Пусть рассматривается система вида
$$
	A\ddot q = -\frac{\partial\Pi(q)}{\partial q},
$$
	где $A = A\T$, $\Pi(0) = \frac{\partial\Pi}{\partial q}(0) = 0$, $(q,\,\dot q) = (0,\,0)$~--- изолированное положение равновесия системы.

	Тогда если $q = 0$ не является точкой локального минимума функции~$\Pi(q)$, причём существует отрицательное собственне значение матрицы~$\frac{\partial^2\Pi}{\partial q^2}(0)$, то положение~$(0,\,0)$ не устойчиво по Ляпунову.
\end{theorem}

Действительно, для нашей системы~\eqref{eq:stability_1} существует функция
$$
	\Pi(\varphi) = \frac{\xi}{2ml}\varphi^2 + \frac{g}{l}\cos\varphi - \frac{g}{l}
$$
такая, что выполнены условия предыдущей теоремы. Единственное собственное значение Якобиана $\frac{\partial^2\Pi}{\partial q^2}(0)$ вычисляется по формуле
$$
	\lambda = \frac{\xi}{ml} - \frac{g}{l}.
$$
Получается, что при
\begin{equation}\label{eq:stability_2}
	\xi < mg.
\end{equation}
верхнее положение равновесия ``перевёрнутого'' маятника будет неустойчиво. Это соотносится с физикой рассматриваемой системы. Далее мы будем рассматривать систему при условии~\eqref{eq:stability_2}.