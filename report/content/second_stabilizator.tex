\subsection{Линейный стабилизатор по динамической обратной связи}

Пусть наряду с системой~\eqref{eq:linear_system} нам задано уравнение наблюдателя~(или измерителя):
\begin{equation}\label{eq:estimator}
	y = Cz,
\end{equation}
где $y$~--- известная вектор функция, интерпретируемая как показания приборов.
Мы считаем, что в каждый момент времени нам известно лишь положение тележки и груза, но не их скорости, поэтому в нашем случае
$$
	C = \begin{pmatrix}
		1 & 0 & 0 & 0 \\
		0 & 0 & 1 & 0
	\end{pmatrix}.
$$
Требуется построить такую оценку $\hat z(t)$ вектора состояния $z(t)$, чтобы она обладала свойством
$$
	\hat z(t) - z(t) \xrightarrow[t \to \infty]{} 0.
$$
Тогда, согласно теории, стабилизирующее управление системой~\eqref{eq:linear_system} примет вид
$$
	u = K \hat z,
$$
а асимптотическим идентификатором состояния линейной системы~\eqref{eq:linear_system},~\eqref{eq:estimator} является
$$
	\dot{\hat z} = A \hat z + B u + L(y - C \hat z),
$$
где неизвестная матрица $L$ подлежит определению.

Рассмотрим две вспомогательные системы:
\begin{equation}
	\begin{aligned}
\dot z_1 &= A z_1 + B u_1,
\\
\dot z_2 &= A\T z_2 - C\T u_2,
	\end{aligned}
\end{equation}
где переменные $z_1$, $z_2$, $u_1$, $u_2$ носят характер формальных обозначений.
Управляемость первой вспомогательной системы исследована в предыдущем разделе.
Исследуем управляемость второй  вспомогательной системы аналогично.
Матрица управляемости второй вспомогательной системы имеет вид
$$
	[C\T,\,A\T C\T,\, (A\T)^2 C\T] = 
	\begin{pmatrix}
		1 & 0 & 0 & 0 & \frac{mg - \xi}{lm} & 0
		\\
		0 & 0 & 1 & 0 & 0 & 0
		\\
		0 & 1 & 0 & 0 & \frac{\xi - mg}{lm} & 0
		\\
		0 & 0 & 0 & 1 & 0 & -\frac{k}{M}
	\end{pmatrix},
$$
и очевидно имеет ранг равный размерности системы. Это значит, что вторая вспомогательная система так же полностью управляема.

Тогда мы можем найти искомые матрицы $K$ и $L\T$ как решение задачи линейной стабилизации первой и второй вспомогательных систем соответственно.
Для первой системы решение известно из предыдущего раздела, поэтому далее будем искать матрицу $L$.

Пусть $-C\T = [q_1, q_2]$, тогда составим матрицу $T = [e_1, e_2, e_3, e_4], \mathrm{det}\,T \neq 0$, где
$$
	\begin{aligned}
	&e_1 = q_1         = \begin{pmatrix}-1\\0\\0\\0\end{pmatrix},
	&&e_2 = A\T q_1     = \begin{pmatrix}0\\-1\\0\\0\end{pmatrix},
	\\
	&e_3 = (A\T)^2 q_1 = \begin{pmatrix}-\frac{mg-\xi}{ml}\\0\\\frac{mg-\xi}{ml}\\0\end{pmatrix},
	&&e_4 = (A\T)^3 q_1 = \begin{pmatrix}0\\-\frac{mg-\xi}{ml}\\0\\\frac{mg-\xi}{ml}\end{pmatrix}.&
	\end{aligned}
$$
Делаем замену переменных $z = T\zeta$. Тогда
$$
	\tilde A = T^{-1}A\T T = \begin{pmatrix}
		0 & 0 & 0 & 0 \\
		1 & 0 & 0 & -\frac{k(\xi - mg)}{Mlm} \\
		0 & 1 & 0 & -\frac{\xi - mg}{lm} \\
		0 & 0 & 1 & -\frac{k}{M}
	\end{pmatrix}
$$
и замечаем, что полученная матрица имеет форму Фробениуса. Это означает, что коэффициенты ее характеристического полинома равны соответственно:
$$
	p_1 = \frac{k}{M},\; p_2 = \frac{\xi - mg}{lm},\; p_3 = \frac{k(\xi - mg)}{Mlm}, \;p_4 = 0.
$$
Пусть $d_i$ --- коэффициенты эталонного полинома:
$$
	\psi(\lambda) = \prod_{i = 1}^{n} (\lambda - \nu_i),
$$
тогда можем построить матрицу
$$
	L\T = \Gamma(T\tilde K)^{-1},
$$
где
$$
	\Gamma = \begin{pmatrix}p_1 - d_1 & p_2 - d_2 & p_3 - d_3 & p_4 - d_4 \\
		0 & 0 & 0 & 0
		\end{pmatrix},\;
	\tilde K = \begin{pmatrix}1 & p_1 & p_2 & p_3 \\
		0 & 1 & p_1 & p_2 \\
		0 & 0 & 1 & p_1
	\end{pmatrix}.
$$