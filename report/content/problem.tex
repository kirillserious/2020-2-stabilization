\section{Постановка задачи}

Рассматривается математическая модель движения вертикального маятника на подвижном основании (тележке) массы $M$.
На конце маятника расположен груз малых размеров массы $m$.
Длина невесомого стержня маятника равна $l$.
Предполагается, что маятник может совершать вращательные движения относительно его крепления без ограничений на углы отклонения.
К шарниру, соединяющему маятник и тележку, прикреплена спиральная пружина жесткости $\xi$.
При верхнем, вертикальном положении маятника пружина не деформирована.
Сила сопротивления пружины, действующая на маятник, пропорциональна величине угла, отсчитываемого от вертикального положения маятника. Тележка может двигаться по прямой (в горизонтальной плоскости). Задан коэффициент вязкого трения тележки о воздух (сила трения пропорциональна первой степени скорости)~--- $k$. Силой трения, действующей на маятник и груз, можно пренебречь. К тележке может быть приложена сила $u \in \setR$, позволяющая двигаться влево или вправо.

Необходимо последовательно выполнить следующие задания:
\begin{enumerate}
	\item
Выписать математические уравнения системы.
Определить параметры, соответствующие верхнему, неустойчивому положению равновесия маятника и неподвижному положению тележки в заданной точке с координатами $s^*$.
Обосновать неустойчивость.
	\item
Выписать математические уравнения для ``линеаризованной'' модели движения в малой окрестности найденного неустойчивого положения равновесия.
	\item
Написать программу, рассчитывающую траектории движения исходной, нелинейной системы при заданном законе управления (в позиционной форме или в зависимости от вектора наблюдений).
	\item
Исследовать возможность построения линейного стабилизатора для линеаризованной системы, делающего найденное неустойчивое положение равновесия асимптотически устойчивым:
	\begin{enumerate}
		\item
линейный стабилизатор по полной обратной связи с заданными собственными значениями замкнутой системы;
		\item
линейный стабилизатор по динамической обратной связи с заданными собственными значениями замкнутой системы и коэффициентами усиления асимптотического наблюдателя (при условии, что в каждый момент времени доступны только координаты тел и конца маятника, но не их скорости);
		\item
линейный стабилизатор, решающий линейно-квадратичную задачу оптимальной стабилизации (при известной полной позиции системы), с заданными матрицами из функционала качества.
	\end{enumerate}
\end{enumerate}

Для каждого случая написать программу, позволяющую построить линейный стабилизатор по произвольным заданным пользователем параметрам (собственным значениям замкнутой системы; матрицам из функционала качества).
Построенный стабилизатор должен быть подставлен, как в линеаризованную, так и в исходную нелинейную систему.
Необходимо поставить эксперименты относительно возможности использования построенного стабилизатора при различных отклонениях начального положения от равновесного, а также при случайных малых возмущениях, действующих на маятник (верхнюю точку).