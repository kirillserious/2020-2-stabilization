\subsection{Линеаризация модели}

Разложив в ряд Тейлора синусы и косинусы в уравнении~\eqref{eq:modeling_equation} в окрестности интересующего нас положения $\varphi = 0$, получим следующую линеаризацию:
$$
	\ddot \varphi + \frac{\xi - mg}{ml} \varphi + \frac{1}{l} \ddot s = 0.
$$

Для упрощения дальнейших рассуждений мы введем новый вектор состояний, первая компонента которого будет представлять аппроксимацию перемещения груза, то есть
$$
	z(t) = [s(t) + l\varphi(t),\,\dot s(t) + l\dot\varphi(t),\,s(t),\,\dot s(t)]\T.
$$
Таким образом получили линеаризованную систему:
\begin{equation}\label{eq:linear_system}
	\left\{
	\begin{aligned}
\dot z_1 &= z_2
\\
\dot z_2 &= \frac{mg - \xi}{ml} (z_1 - z_3)
\\
\dot z_3 &= z_4
\\
\dot z_4 &= \frac{u}{M} - \frac{k}{M}z_4.
	\end{aligned}
	\right.
	\Longleftrightarrow
	\dot z =
	\begin{pmatrix}
0 & 1 & 0 & 0
\\
\frac{mg - \xi}{ml} & 0 & -\frac{mg - \xi}{ml} & 0
\\
0 & 0 & 0 & 1
\\
0 & 0 & 0 & -\frac{k}{M}
	\end{pmatrix}
	z
	+
	\begin{pmatrix}
0 \\ 0 \\ 0 \\ \frac{1}{M}
	\end{pmatrix}
	u.
\end{equation}

Рассмотрим вопрос об управляемости получившейся системы. Воспользуемся следующей теоремой.
\begin{theorem}[Уточнённый критерий Калмана]
	Система $\dot x = Ax + Bu$, $A \in \setR^{n\times n}$, $B \in \setR^{n \times m}$ является полностью управляемой тогда и только тогда, когда ранг матрицы управляемости $C = [B,\,AB,\,\ldots,\,A^{n-m}B]$ равен $n$.
\end{theorem}
Составим матрицу управляемости для линеаризованной системы и посчитаем её определитель.
$$
	C = [B,\,AB,\,A^2 B,\, A^3B] = 
	\begin{pmatrix}
0 & 0 & 0 & \frac{\xi - mg}{M m l}
\\
0 & 0 & \frac{\xi - mg}{M m l} & -k\frac{\xi - mg}{M^2ml}
\\
0 & \frac{1}{M} & -\frac{k}{M^2} & \frac{k^2}{M^3}
\\
\frac{1}{M} & -\frac{k}{M^2} & \frac{k^2}{M^3} & -\frac{k^3}{M^4}
	\end{pmatrix},
$$
$$
	\mathrm{det}\,C = \frac{(\xi - mg)^2}{M^4m^2l^2}.
$$
Получается, если $\xi \neq mg$, то $\mathrm{rank}\,C = n$, а, значит, линеаризованная система полностью управляема.