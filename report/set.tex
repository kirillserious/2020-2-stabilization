% Необходимые пакеты для компиляции русского языка, картинок и прочего

\usepackage[utf8]{inputenc}                % Кодировка
\usepackage[main=russian, english]{babel}  % Русский язык
\usepackage[pdftex]{graphicx}              % Картинки
\usepackage{indentfirst}                   % Отступ перед абзацами

\usepackage{amsmath}  % Математические 
\usepackage{amssymb}  % формулы

\usepackage{tikz}                % Векторная графика
                                 %
\usepackage{pgfplots}            % % Нужно для вставки графиков из 
\pgfplotsset{compat=newest}      % % matlab2tikz
\usetikzlibrary{plotmarks}       % %
\usetikzlibrary{arrows.meta}     % %
\usepgfplotslibrary{patchplots}  % %
\usepackage{grffile}             % %

\usepackage{caption} % Чтобы можно было вставлять формулы к подписям рисунков

\usepackage[unicode]{hyperref}                                         % Ссылки и русские закладки
\hypersetup {                                                          %
    pdftitle={Лабораторная работа},                                    % Название документа
    pdfsubject={Теория стабилизации},                                  % Тема документа
    pdfauthor={Егоров Кирилл Юлианович},                 % Автор документа
    pdfcreator={Кафедра системного анализа ВМК МГУ},     % Создатель документа
    pdfproducer={LaTeX},                                 % Программа, создавшая документ
    hidelinks                                            % Скрывает рамку вокруг ссылок
}


\usepackage{nicefrac}
\usepackage{amsthm}  % Красивый внешний вид теорем, определений и доказательств
\newtheoremstyle{def}
        {\topsep}
        {\topsep}
        {\normalfont}
        {\parindent}
        {\bfseries}
        {.}
        {.5em}
        {}
\theoremstyle{def}
\newtheorem{definition}{Определение}
\newtheorem{example}{Пример}

\newtheoremstyle{th}
        {\topsep}
        {\topsep}
        {\itshape}
        {\parindent}
        {\bfseries}
        {.}
        {.5em}
        {}
\theoremstyle{th}
\newtheorem{theorem}{Теорема}
\newtheorem{lemma}{Лемма}
\newtheorem{assertion}{Утверждение}

\newtheoremstyle{rem}
        {0.5\topsep}
        {0.5\topsep}
        {\normalfont}
        {\parindent}
        {\itshape}
        {.}
        {.5em}
        {}
\theoremstyle{rem}
\newtheorem{remark}{Замечание}

% Новое доказательство
\renewenvironment{proof}{\parД о к а з а т е л ь с т в о.}{\hfill$\blacksquare$}